\documentclass[10pt,a4paper]{article}
%\usepackage{zh\_CN-Adobefonts\_external}
\usepackage{xeCJK}
\usepackage{amsmath, amsthm}
\usepackage{listings,xcolor}
\usepackage{geometry} % 设置页边距
\usepackage{fontspec}
% \usepackage{graphicx}
\usepackage[colorlinks, linkcolor=black]{hyperref}
% \usepackage{setspace}
% \usepackage{pxfonts}
\usepackage{fancyhdr} % 自定义页眉页脚


\setsansfont{Monaco} % 设置英文字体
% \setmonofont[Mapping={}]{Consolas} % 英文引号之类的正常显示,相当于设置英文字体
% \setCJKmainfont{kai}  %中文字体设置

\linespread{1.2}

% \title{Template For ICPC}
% \author{ChenJr @ GDUT}
% \definecolor{dkgreen}{rgb}{0,0.6,0}
% \definecolor{gray}{rgb}{0.5,0.5,0.5}
% \definecolor{mauve}{rgb}{0.58,0,0.82}

\pagestyle{fancy}

\lhead{Xidian University} %以下分别为左中右的页眉和页脚
\chead{}

% \rhead{\CJKfamily{kai} 第 \thepage 页}

\lfoot{} 
\cfoot{\thepage}
\rfoot{}

\renewcommand{\headrulewidth}{0.4pt} 
% \renewcommand{\footrulewidth}{0.4pt}

% \geometry{left=2.5cm,right=3cm,top=2.5cm,bottom=2.5cm} % 页边距
\geometry{left=3.18cm,right=3.18cm,top=2.54cm,bottom=2.54cm}
% \setlength{\columnsep}{30pt}

% \makeatletter

% \makeatother



\lstset{
    language    = c++,
    numbers     = left,
    rulesepcolor= \color{gray},
    breaklines=true,
    numberstyle={                               % 设置行号格式
        \small
        \color{gray}
        % \fontspec{Consolas}
    },
	% commentstyle = \color[RGB]{0,128,0}\bfseries, %代码注释的颜色
	commentstyle = \color{gray},
	keywordstyle={                              % 设置关键字格式
        \color[RGB]{175,0,175}
        % \fontspec{Consolas Bold}
        \bfseries
        % \bold
    },
	stringstyle={                               % 设置字符串格式
        \color[RGB]{0,125,0}
        % \fontspec{Consolas}
        \bfseries
    },
	basicstyle={                                % 设置代码格式
        % \fontspec{Consolas}
        \small\ttfamily
    },
    morekeywords={alignas,continute,friend,register,true,alignof,decltype,goto,
    reinterpret\_cast,try,asm,defult,if,return,typedef,auto,delete,inline,short,
    typeid,bool,do,int,signed,typename,break,double,long,sizeof,union,case,
    dynamic\_cast,mutable,static,unsigned,catch,else,namespace,static\_assert,using,
    char,enum,new,static\_cast,virtual,char16\_t,char32\_t,explict,noexcept,struct,
    void,export,nullptr,switch,volatile,class,extern,operator,template,wchar\_t,
    const,false,private,this,while,constexpr,float,protected,thread\_local,
    const\_cast,for,public,throw,std},
    emph={map,set,multimap,multiset,unordered\_map,unordered\_set,
    unordered\_multiset,unordered\_multimap,vector,string,list,deque,
    array,stack,forwared\_list,iostream,memory,shared\_ptr,unique\_ptr,
    random,bitset,ostream,istream,cout,cin,endl,move,default\_random\_engine,
    uniform\_int\_distribution,iterator,algorithm,functional,bing,numeric,},
	emphstyle=\color[RGB]{112,64,160},          % 设置强调字格式
    breaklines=true,                            % 设置自动换行
    tabsize     = 4,
    frame       = single,%主题
    columns     = fullflexible,
    rulesepcolor = \color{red!20!green!20!blue!20}, %设置边框的颜色
    showstringspaces = false, %不显示代码字符串中间的空格标记
	escapeinside={\%*}{*)},
}

\begin{document}
\title{ICPC Templates}
\author {Leachim}
\maketitle
\tableofcontents

\newpage
\section{图论}
\subsection{最小生成树Kruskal}
\lstinputlisting{../图论/最小生成树Kruskal.cpp}
\subsection{最小生成树Prim}
\lstinputlisting{../图论/最小生成树Prim.cpp}
\subsection{最短路Djikstra}
\lstinputlisting{../图论/最短路Djikstra.cpp}
\subsection{最短路SPFA}
\lstinputlisting{../图论/最短路SPFA.cpp}
\subsection{最近公共祖先LCA\_Doubling}
\lstinputlisting{../图论/最近公共祖先LCA_Doubling.cpp}
\subsection{最近公共祖先LCA\_Tarjan}
\lstinputlisting{../图论/最近公共祖先LCA_Tarjan.cpp}
\subsection{判断负环SPFA\_Negtive\_Cycle}
\lstinputlisting{../图论/判断负环SPFA_Negtive_Cycle.cpp}
\subsection{拓扑排序Topological\_Sort\_Khan}
\lstinputlisting{../图论/拓扑排序Topological_Sort_Khan.cpp}

\newpage
\section{多项式}
\subsection{FFT字符串匹配String\_Match\_FFT}
\lstinputlisting{../多项式/FFT字符串匹配String_Match_FFT.cpp}
\subsection{FFT递归Fast\_Fourier\_Transform\_Cooley-Tukey\_Recursion}
\lstinputlisting{../多项式/FFT递归Fast_Fourier_Transform_Cooley-Tukey_Recursion.cpp}
\subsection{FFT递推Fast\_Fourier\_Transform\_Cooley-Tukey\_Iteration}
\lstinputlisting{../多项式/FFT递推Fast_Fourier_Transform_Cooley-Tukey_Iteration.cpp}
\subsection{NTT递推Number\_Theoretic\_Transforms}
\lstinputlisting{../多项式/NTT递推Number_Theoretic_Transforms.cpp}
\subsection{多项式求逆Polynomial\_Inverse}
\lstinputlisting{../多项式/多项式求逆Polynomial_Inverse.cpp}

\newpage
\section{字符串}
\subsection{字符串哈希String\_Hash}
\lstinputlisting{../字符串/字符串哈希String_Hash.cpp}
\subsection{马拉车Manacher}
\lstinputlisting{../字符串/马拉车Manacher.cpp}
\subsection{字符串匹配KMP}
\lstinputlisting{../字符串/字符串匹配KMP.cpp}
\subsection{AC自动机AC-Automaton}
\lstinputlisting{../字符串/AC自动机AC-Automaton.cpp}
\subsection{后缀数组Suffix\_Array}
\lstinputlisting{../字符串/后缀数组Suffix_Array.cpp}
\subsection{后缀自动机Suffix-Automaton}
\lstinputlisting{../字符串/后缀自动机Suffix-Automaton.cpp}
\subsection{广义后缀自动机General\_Suffix-Automaton}
\lstinputlisting{../字符串/广义后缀自动机General_Suffix-Automaton.cpp}

\newpage
\section{数据结构}
\subsection{并查集Union\_Find}
\lstinputlisting{../数据结构/并查集Union_Find.cpp}
\subsection{分块Block\_1}
\lstinputlisting{../数据结构/分块Block_1.cpp}
\subsection{分块Block\_2}
\lstinputlisting{../数据结构/分块Block_2.cpp}
\subsection{分块Block\_4}
\lstinputlisting{../数据结构/分块Block_4.cpp}
\subsection{树状数组Binary\_Indexed\_Tree}
\lstinputlisting{../数据结构/树状数组Binary_Indexed_Tree.cpp}
\subsection{树状数组2D\_Binary\_Indexed\_Tree}
\lstinputlisting{../数据结构/树状数组2D_Binary_Indexed_Tree.cpp}
\subsection{线段树Segment\_Tree}
\lstinputlisting{../数据结构/线段树Segment_Tree.cpp}
\subsection{线段树Segment\_Tree\_Multiply}
\lstinputlisting{../数据结构/线段树Segment_Tree_Multiply.cpp}
\subsection{扫描线Scanline}
\lstinputlisting{../数据结构/扫描线Scanline.cpp}
\subsection{zkw线段树ZKW\_Segment\_Tree}
\lstinputlisting{../数据结构/zkw线段树ZKW_Segment_Tree.cpp}
\subsection{李超线段树Li-Chao\_Segment\_Tree}
\lstinputlisting{../数据结构/李超线段树Li-Chao_Segment_Tree.cpp}
\subsection{可并堆左偏树Leftist\_Tree}
\lstinputlisting{../数据结构/可并堆左偏树Leftist_Tree.cpp}
\subsection{Splay树Splay\_Tree}
\lstinputlisting{../数据结构/Splay树Splay_Tree.cpp}
\subsection{Splay树Splay\_Tree\_Flip}
\lstinputlisting{../数据结构/Splay树Splay_Tree_Flip.cpp}
\subsection{Splay树Splay\_Tree\_Dye\&Flip}
\lstinputlisting{../数据结构/Splay树Splay_Tree_Dye&Flip.cpp}
% \subsection{红黑树Red\_Black\_Tree}
% \lstinputlisting{../数据结构/红黑树Red_Black_Tree.cpp}

\newpage
\section{数论}
\subsection{乘法逆元Multiplicative\_Inverse\_Modulo}
\lstinputlisting{../数论/乘法逆元Multiplicative_Inverse_Modulo.cpp}
\subsection{卢卡斯Lucas}
\lstinputlisting{../数论/卢卡斯Lucas.cpp}
\subsection{拓展欧几里得Exgcd}
\lstinputlisting{../数论/拓展欧几里得Exgcd.cpp}
\subsection{拓展欧拉定理Ex\_Euler\_Theorem-Automaton}
\lstinputlisting{../数论/拓展欧拉定理Ex_Euler_Theorem.cpp}
\subsection{欧拉筛Eular\_Sieve}
\lstinputlisting{../数论/欧拉筛Eular_Sieve.cpp}
\subsection{杜教筛Dujiao\_Sieve}
\lstinputlisting{../数论/杜教筛Dujiao_Sieve.cpp}
\subsection{求原根Get\_Primitive\_Root}
\lstinputlisting{../数论/求原根Get_Primitive_Root.cpp}
\subsection{贝祖引理Bezout\_Lemma}
\lstinputlisting{../数论/贝祖引理Bezout_Lemma.cpp}
\subsection{除法分块Block\_Division}
\lstinputlisting{../数论/除法分块Block_Division.cpp}

\newpage
\section{网络流}
\subsection{最大费用流Minimum-Cost\_Flow\_Edmonds-Karp}
\lstinputlisting{../网络流/最大费用流Minimum-Cost_Flow_Edmonds-Karp.cpp}
\subsection{最大流Maximum\_Flow\_Edmonds-Karp}
\lstinputlisting{../网络流/最大流Maximum_Flow_Edmonds-Karp.cpp}
\subsection{最大流Maximum\_Flow\_Dinic}
\lstinputlisting{../网络流/最大流Maximum_Flow_Dinic.cpp}
\subsection{二分题最大匹配Bipartite\_Graph\_Maximum\_Matching\_Dinic}
\lstinputlisting{../网络流/二分题最大匹配Bipartite_Graph_Maximum_Matching_Dinic.cpp}
\subsection{二分图最大匹配Bipartite\_Graph\_Maximum\_Matching\_Hungarian}
\lstinputlisting{../网络流/二分图最大匹配Bipartite_Graph_Maximum_Matching_Hungarian.cpp}
\subsection{二分图最大匹配Bipartite\_Graph\_Maximum\_Matching\_Hopcroft-Karp}
\lstinputlisting{../网络流/二分图最大匹配Bipartite_Graph_Maximum_Matching_Hopcroft-Karp.cpp}

\newpage
\section{计算几何}
\subsection{计算几何Compuational\_Geometry}
\lstinputlisting{../计算几何/计算几何Compuational_Geometry.cpp}



\end{document}