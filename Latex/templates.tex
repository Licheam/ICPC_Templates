% !TEX TS-program = pdflatexmk
\documentclass[10pt,a4paper]{ctexart}
%\usepackage{zh\_CN-Adobefonts\_external}
\usepackage{xeCJK}
\usepackage{amsmath, amsthm}
\usepackage{listings,xcolor}
\usepackage{geometry} % 设置页边距
\usepackage{fontspec}
\usepackage{graphicx}
\usepackage[colorlinks, linkcolor=black]{hyperref}
\usepackage{xcolor}
% \usepackage{setspace}
% \usepackage{pxfonts}
\usepackage{fancyhdr} % 自定义页眉页脚
\usepackage{ctex}

\usepackage{relsize}
% \renewcommand{\_}{\textscale{.5}{\textunderscore}}
\renewcommand{\_}{\textscale{1}{\textunderscore}}

\setmainfont{Monaco}
\setmonofont{Monaco}
\setsansfont{Monaco} % 设置英文字体
% \setmonofont[Mapping={}]{Consolas} % 英文引号之类的正常显示,相当于设置英文字体
% \setCJKmainfont{kai}  %中文字体设置
\setCJKmainfont{Kai}

\ctexset{
    section={
        name={,、},
        number={\chinese{section}},
        format={\zihao{3}\centering},
        aftername={}
    }
}
\ctexset{
    subsection={
        format={\zihao{-4}},
        aftername={ }
    }
}
\ctexset{
    subsubsection={
        format={\zihao{-4}},
        aftername={ }
    }
}

\linespread{1.2}

% \title{Template For ICPC}
% \author{ChenJr @ GDUT}
% \definecolor{dkgreen}{rgb}{0,0.6,0}
% \definecolor{gray}{rgb}{0.5,0.5,0.5}
% \definecolor{mauve}{rgb}{0.58,0,0.82}

\pagestyle{fancy}

% \lhead{Xidian University} %以下分别为左中右的页眉和页脚
% \chead{Leachim}

% \rhead{\CJKfamily{kai} 第 \thepage 页}

% \lfoot{西安电子科技大学} 
% \cfoot{\thepage}
% \rfoot{}

\renewcommand{\headrulewidth}{0.4pt} 
% \renewcommand{\footrulewidth}{0.4pt}

% \geometry{left=2.5cm,right=2.5cm,top=2.5cm,bottom=2.5cm} % 页边距
\geometry{left=3.18cm,right=3.18cm,top=2.54cm,bottom=2.54cm}
% \setlength{\columnsep}{30pt}

% \makeatletter

% \makeatother

\usepackage{textcomp}


\lstset{
    % inputencoding = utf8,
    % escapebegin=\begin{CJK*}{GBK}{Hei},escapeend=\end{CJK*},
    language    = c++,
    numbers     = left,
    rulesepcolor= \color{gray},
    breaklines=true,
    numberstyle={                               % 设置行号格式
        \small
        \color{gray}
        % \fontspec{Menlo Regular}
    },
	% commentstyle = \color[RGB]{0,128,0}\bfseries, %代码注释的颜色
	commentstyle = \color{gray},
	keywordstyle={                              % 设置关键字格式
        \color[RGB]{175,50,175}
        % \color{magenta}
        % \fontspec{Menlo Regular}
        \bfseries
        % \bold
    },
	stringstyle={                               % 设置字符串格式
        \color[RGB]{0,125,0}
        % \fontspec{Menlo Regular}
        \bfseries
    },
	basicstyle={                                % 设置代码格式
        % \fontspec{Menlo}
        \small\ttfamily
    },
    % identifierstyle=\color{black},
 %    morekeywords={alignas,continute,friend,register,true,alignof,decltype,goto,
 %    reinterpret\_cast,try,asm,defult,if,return,typedef,auto,delete,inline,short,
 %    typeid,bool,do,int,signed,typename,break,double,long,sizeof,union,case,
 %    dynamic\_cast,mutable,static,unsigned,catch,else,namespace,static\_assert,using,
 %    char,enum,new,static\_cast,virtual,char16\_t,char32\_t,explict,noexcept,struct,
 %    void,export,nullptr,switch,volatile,class,extern,operator,template,wchar\_t,
 %    const,false,private,this,while,constexpr,float,protected,thread\_local,
 %    const\_cast,for,public,throw,std},
 %    emph={map,set,multimap,multiset,unordered\_map,unordered\_set,
 %    unordered\_multiset,unordered\_multimap,vector,string,list,deque,
 %    array,stack,forwared\_list,iostream,memory,shared\_ptr,unique\_ptr,
 %    random,bitset,ostream,istream,cout,cin,endl,move,default\_random\_engine,
 %    uniform\_int\_distribution,iterator,algorithm,functional,bing,numeric,},
	% emphstyle=\color[RGB]{112,64,160},          % 设置强调字格式
    breaklines=true,                            % 设置自动换行
    tabsize     = 4,
    frame       = single,%主题
    columns     = fullflexible,
    % rulesepcolor = \color{red!20!green!20!blue!20}, %设置边框的颜色
    rulesepcolor = \color{gray},
    showstringspaces = false, %不显示代码字符串中间的空格标记
	escapeinside={\%*}{*)},
}


\usepackage{minted}
\renewcommand{\theFancyVerbLine}{\sffamily \textcolor[RGB]{128,128,128}{\small\arabic{FancyVerbLine}}}
% \renewcommand{\theFancyVerbLine}{\textcolor[RGB]{211,211,211}\sffamily\small\arabic{FancyVerbLine}}
\setminted[c++]{
    breaklines=true,
    breakautoindent=false,
    breakanywhere=true,
    % escapeinside=\#\%,
    mathescape=true,
    linenos=true,
    tabsize=4,
    frame=single,
    style=arduino
}

\usepackage{relsize}
% \renewcommand{\_}{\textscale{.5}{\textunderscore}}
\renewcommand{\_}{\textscale{1}{\textunderscore}}

\begin{document}
\title{\zihao{0}{ICPC Templates}}
\author {\zihao{2}Leachim}
\maketitle

\begin{figure}[htbp]
    \centering{\includegraphics[width=10cm]{../Figures/ICPC.png}}
\end{figure}

\newpage

\tableofcontents

\newpage
\section{图论}
\subsection{树的重心Get\_Centroid}
\inputminted{c++}{../图论/树的重心Get_Centroid.cpp}
\subsection{最小生成树Kruskal}
\inputminted{c++}{../图论/最小生成树Kruskal.cpp}
\subsection{最小生成树Prim}
\inputminted{c++}{../图论/最小生成树Prim.cpp}
\subsection{最短路Djikstra}
\inputminted{c++}{../图论/最短路Djikstra.cpp}
\subsection{最短路SPFA}
\inputminted{c++}{../图论/最短路SPFA.cpp}
\subsection{最近公共祖先LCA\_Doubling}
\inputminted{c++}{../图论/最近公共祖先LCA_Doubling.cpp}
\subsection{最近公共祖先LCA\_Tarjan}
\inputminted{c++}{../图论/最近公共祖先LCA_Tarjan.cpp}
\subsection{最近公共祖先LCA\_Heavy-Light\_Decomposition}
\inputminted{c++}{../图论/最近公共祖先LCA_Heavy-Light_Decomposition.cpp}
\subsection{判断负环SPFA\_Negtive\_Cycle}
\inputminted{c++}{../图论/判断负环SPFA_Negtive_Cycle.cpp}
\subsection{拓扑排序Topological\_Sort\_Khan}
\inputminted{c++}{../图论/拓扑排序Topological_Sort_Khan.cpp}
\subsection{树链剖分Heavy-Light\_Decomposition}
\inputminted{c++}{../图论/树链剖分Heavy-Light_Decomposition.cpp}

\newpage
\section{多项式}
\subsection{FFT字符串匹配String\_Match\_FFT}
\inputminted{c++}{../多项式/FFT字符串匹配String_Match_FFT.cpp}
\subsection{FFT递归Fast\_Fourier\_Transform\_Cooley-Tukey\_Recursion}
\inputminted{c++}{../多项式/FFT递归Fast_Fourier_Transform_Cooley-Tukey_Recursion.cpp}
\subsection{FFT递推Fast\_Fourier\_Transform\_Cooley-Tukey\_Iteration}
\inputminted{c++}{../多项式/FFT递推Fast_Fourier_Transform_Cooley-Tukey_Iteration.cpp}
\subsection{NTT递推Number\_Theoretic\_Transforms}
\inputminted{c++}{../多项式/NTT递推Number_Theoretic_Transforms.cpp}
\subsection{MTT任意模数Fast\_Fourier\_Transform\_MOD}
\inputminted{c++}{../多项式/MTT任意模数Fast_Fourier_Transform_MOD.cpp}
\subsection{多项式求逆Polynomial\_Inverse}
\inputminted{c++}{../多项式/多项式求逆Polynomial_Inverse.cpp}

\newpage
\section{字符串}
\subsection{字符串哈希String\_Hash}
\inputminted{c++}{../字符串/字符串哈希String_Hash.cpp}
\subsection{最小表示法Lexicographically\_Minimal\_String\_Rotation}
\inputminted{c++}{../字符串/最小表示法Lexicographically_Minimal_String_Rotation.cpp}
\subsection{马拉车Manacher}
\inputminted{c++}{../字符串/马拉车Manacher.cpp}
\subsection{字符串匹配KMP}
\inputminted{c++}{../字符串/字符串匹配KMP.cpp}
\subsection{AC自动机AC-Automaton}
\inputminted{c++}{../字符串/AC自动机AC-Automaton.cpp}
\subsection{后缀数组Suffix\_Array}
\inputminted{c++}{../字符串/后缀数组Suffix_Array.cpp}
\subsection{后缀自动机Suffix-Automaton}
\inputminted{c++}{../字符串/后缀自动机Suffix-Automaton.cpp}
\subsection{广义后缀自动机General\_Suffix-Automaton}
\inputminted{c++}{../字符串/广义后缀自动机General_Suffix-Automaton.cpp}

\newpage
\section{数据结构}
\subsection{并查集Disjoin\_Set\_Union}
\inputminted{c++}{../数据结构/并查集Disjoin_Set_Union.cpp}
\subsection{可撤销并查集Disjoin\_Set\_Union\_Withdrawable}
\inputminted{c++}{../数据结构/可撤销并查集Disjoin_Set_Union_Withdrawable.cpp}
\subsection{分块Block\_1}
\inputminted{c++}{../数据结构/分块Block_1.cpp}
\subsection{分块Block\_2}
\inputminted{c++}{../数据结构/分块Block_2.cpp}
\subsection{分块Block\_4}
\inputminted{c++}{../数据结构/分块Block_4.cpp}
\subsection{树状数组Binary\_Indexed\_Tree}
\inputminted{c++}{../数据结构/树状数组Binary_Indexed_Tree.cpp}
\subsection{二维树状数组2D\_Binary\_Indexed\_Tree}
\inputminted{c++}{../数据结构/二维树状数组2D_Binary_Indexed_Tree.cpp}
\subsection{线段树Segment\_Tree}
\inputminted{c++}{../数据结构/线段树Segment_Tree.cpp}
\subsection{线段树Segment\_Tree\_Multiply}
\inputminted{c++}{../数据结构/线段树Segment_Tree_Multiply.cpp}
\subsection{扫描线Scanline}
\inputminted{c++}{../数据结构/扫描线Scanline.cpp}
\subsection{zkw线段树ZKW\_Segment\_Tree}
\inputminted{c++}{../数据结构/zkw线段树ZKW_Segment_Tree.cpp}
\subsection{李超线段树Li-Chao\_Segment\_Tree}
\inputminted{c++}{../数据结构/李超线段树Li-Chao_Segment_Tree.cpp}
\subsection{可并堆左偏树Leftist\_Tree}
\inputminted{c++}{../数据结构/可并堆左偏树Leftist_Tree.cpp}
\subsection{Splay树Splay\_Tree}
\inputminted{c++}{../数据结构/Splay树Splay_Tree.cpp}
\subsection{Splay树Splay\_Tree\_Flip}
\inputminted{c++}{../数据结构/Splay树Splay_Tree_Flip.cpp}
\subsection{Splay树Splay\_Tree\_Dye\&Flip}
\inputminted{c++}{../数据结构/Splay树Splay_Tree_Dye\&Flip.cpp}
\subsection{动态树Link-Cut\_Tree}
\inputminted{c++}{../数据结构/动态树Link-Cut_Tree.cpp}
\subsection{红黑树Red\_Black\_Tree}
\inputminted{c++}{../数据结构/红黑树Red_Black_Tree.cpp}

\newpage
\section{数论}
\subsection{线性求逆元Modular\_Multiplicative\_Inverse}
\inputminted{c++}{../数论/线性求逆元Modular_Multiplicative_Inverse.cpp}
\subsection{数论分块Block\_Division}
\inputminted{c++}{../数论/数论分块Block_Division.cpp}
\subsection{贝祖引理Bezout\_Lemma}
\inputminted{c++}{../数论/贝祖引理Bezout_Lemma.cpp}
\subsection{卢卡斯Lucas}
\inputminted{c++}{../数论/卢卡斯Lucas.cpp}
\subsection{拓展欧几里得Exgcd}
\inputminted{c++}{../数论/拓展欧几里得Exgcd.cpp}
\subsection{拓展欧拉定理Ex\_Euler\_Theorem-Automaton}
\inputminted{c++}{../数论/拓展欧拉定理Ex_Euler_Theorem.cpp}
\subsection{中国剩余定理Chinese\_Remainder\_Theorem}
\inputminted{c++}{../数论/中国剩余定理Chinese_Remainder_Theorem.cpp}
\subsection{拓展中国剩余定理Ex\_Chinese\_Remainder\_Theorem}
\inputminted{c++}{../数论/拓展中国剩余定理Ex_Chinese_Remainder_Theorem.cpp}
\subsection{欧拉筛Eular\_Sieve}
\inputminted{c++}{../数论/欧拉筛Eular_Sieve.cpp}
\subsection{杜教筛Dujiao\_Sieve}
\inputminted{c++}{../数论/杜教筛Dujiao_Sieve.cpp}
\subsection{求原根Get\_Primitive\_Root}
\inputminted{c++}{../数论/求原根Get_Primitive_Root.cpp}
\subsection{素数测试Miller\_Rabin}
\inputminted{c++}{../数论/素数测试Miller_Rabin.cpp}
\subsection{大数分解Pollard\_Rho}
\inputminted{c++}{../数论/大数分解Pollard_Rho.cpp}


\newpage
\section{组合数学}
\subsection{康托展开Cantor}
\inputminted{c++}{../组合数学/康托展开Cantor.cpp}
\subsection{波利亚P\'{o}lya}
\inputminted{c++}{../组合数学/波利亚Polya.cpp}
\subsection{卡特兰数Catalan}
\begin{enumerate}
\item 有 $2n$ 个人排成一行进入剧场。入场费 5 元。其中只有 $n$ 个人有一张 5 元钞票,另外 $n$ 人只有 10 元钞票,剧院无其它钞票,问有多少中方法使得只要有 10 元的人买票,售票处就有 5 元的钞票找零?
\item 一位大城市的律师在她住所以北 $n$ 个街区和以东 $n$ 个街区处工作。每天她走 $2n$ 个街区去上班。如果他从不穿越(但可以碰到)从家到办公室的对角线,那么有多少条可能的道路?
\item 在圆上选择 $2n$ 个点,将这些点成对连接起来使得所得到的 $n$ 条线段不相交的方法数?
\item 对角线不相交的情况下,将一个凸多边形区域分成三角形区域的方法数?
\item 一个栈(无穷大)的进栈序列为 $1,2,3, \cdots ,n$ 有多少个不同的出栈序列?
\item $n$ 个结点可构造多少个不同的二叉树?
\item $n$ 个不同的数依次进栈,求不同的出栈结果的种数?
\item $n$ 个 $+1$ 和 $n$ 个 $-1$ 构成 $2n$ 项 $a_1,a_2, \cdots ,a_{2n}$ ,其部分和满足 $a_1+a_2+ \cdots +a_k \geq 0(k=1,2,3, \cdots ,2n)$ 对与 $n$ 该数列为?
\end{enumerate}

\begin{table}[ht]
\begin{tabular}{|l|l|l|l|l|l|l|}
\hline
H0 & H1 & H2 & H3 & H4 & H5 & H6  \\ \hline
1  & 1  & 2  & 5  & 14 & 42 & 132 \\ \hline
\end{tabular}
\end{table}


关于 Catalan 数的常见公式:

$$
H_n = \begin{cases}
    \sum_{i=1}^{n} H_{i-1} H_{n-i} & n \geq 2, n \in \mathbf{N_{+}}\\
    1 & n = 0, 1
\end{cases}
$$

$$
H_n = \frac{H_{n-1} (4n-2)}{n+1}
$$

$$
H_n = \binom{2n}{n} - \binom{2n}{n-1}
$$

$$
H_n = \frac{\binom{2n}{n}}{n+1}(n \geq 2, n \in \mathbf{N_{+}})
$$
\subsection{斯特林数Stirling}
\subsubsection{第一类斯特林数}

(斯特林轮换数)

$\begin{bmatrix}n\\ k\end{bmatrix}$ 表示将 $n$ 个两两不同的元素,划分为 $k$ 个非空圆排列的方案数。

递推式

$$
\begin{bmatrix}n\\ k\end{bmatrix}=\begin{bmatrix}n-1\\ k-1\end{bmatrix}+(n-1)\begin{bmatrix}n-1\\ k\end{bmatrix}
$$

边界是 $\begin{bmatrix}n\\ 0\end{bmatrix}=[n=0]$ 。

\subsubsection{第二类斯特林数}

(斯特林子集数)

$\begin{Bmatrix}n\\ k\end{Bmatrix}$ 表示将 $n$ 个两两不同的元素,划分为 $k$ 个非空子集的方案数。

递推式

$$
\begin{Bmatrix}n\\ k\end{Bmatrix}=\begin{Bmatrix}n-1\\ k-1\end{Bmatrix}+k\begin{Bmatrix}n-1\\ k\end{Bmatrix}
$$

边界是 $\begin{Bmatrix}n\\ 0\end{Bmatrix}=[n=0]$ 。

\subsubsection{上升幂与普通幂的相互转化}

我们记上升阶乘幂 $x^{\overline{n}}=\prod_{k=0}^{n-1} (x+k)$ 。

则可以利用下面的恒等式将上升幂转化为普通幂:

$$
x^{\overline{n}}=\sum_{k} \begin{bmatrix}n\\ k\end{bmatrix} x^k
$$

如果将普通幂转化为上升幂,则有下面的恒等式:

$$
x^n=\sum_{k} \begin{Bmatrix}n\\ k\end{Bmatrix} (-1)^{n-k} x^{\overline{k}}
$$

\subsubsection{下降幂与普通幂的相互转化}

我们记下降阶乘幂 $x^{\underline{n}}=\dfrac{x!}{(x-n)!}=\prod_{k=0}^{n-1} (x-k)$ 。

则可以利用下面的恒等式将普通幂转化为下降幂:

$$
x^n=\sum_{k} \begin{Bmatrix}n\\ k\end{Bmatrix} x^{\underline{k}}
$$

如果将下降幂转化为普通幂,则有下面的恒等式:

$$
x^{\underline{n}}=\sum_{k} \begin{bmatrix}n\\ k\end{bmatrix} (-1)^{n-k} x^k
$$
\subsection{范德蒙德卷积Vandermonde\_Convolution}
$$
\sum\limits_{i=0}^k\binom{n}{i}\binom{m}{k-i}=\binom{n+m}{k}
$$

从数量分别为$n$和$m$的石堆中总共选取$k$个石子。


\newpage
\section{网络流}
\subsection{最大费用流Minimum-Cost\_Flow\_Edmonds-Karp}
\inputminted{c++}{../网络流/最大费用流Minimum-Cost_Flow_Edmonds-Karp.cpp}
\subsection{最大流Maximum\_Flow\_Edmonds-Karp}
\inputminted{c++}{../网络流/最大流Maximum_Flow_Edmonds-Karp.cpp}
\subsection{最大流Maximum\_Flow\_Dinic}
\inputminted{c++}{../网络流/最大流Maximum_Flow_Dinic.cpp}
\subsection{二分图最大匹配Bipartite\_Graph\_Maximum\_Matching\_Dinic}
\inputminted{c++}{../网络流/二分图最大匹配Bipartite_Graph_Maximum_Matching_Dinic.cpp}
\subsection{二分图最大匹配Bipartite\_Graph\_Maximum\_Matching\_Hungarian}
\inputminted{c++}{../网络流/二分图最大匹配Bipartite_Graph_Maximum_Matching_Hungarian.cpp}
\subsection{二分图最大匹配Bipartite\_Graph\_Maximum\_Matching\_Hopcroft-Karp}
\inputminted{c++}{../网络流/二分图最大匹配Bipartite_Graph_Maximum_Matching_Hopcroft-Karp.cpp}

\newpage
\section{计算几何}
\subsection{计算几何Compuational\_Geometry}
\inputminted{c++}{../计算几何/计算几何Compuational_Geometry.cpp}
\subsection{圆的K次面积交K-Intersection\_Area\_of\_Circles}
\inputminted{c++}{../计算几何/圆的K次面积交K-Intersection_Area_of_Circles.cpp}

\newpage
\section{优化算法}
\subsection{wqs二分+决策单调性dp}
\inputminted{c++}{../优化算法/wqs二分+决策单调性dp.cpp}
\subsection{斜率优化Slope\_Optimization}
\inputminted{c++}{../优化算法/斜率优化Slope_Optimization.cpp}

\newpage
\section{离线算法}
\subsection{莫队算法Mo}
\inputminted{c++}{../离线算法/莫队算法Mo.cpp}
\subsection{线段树分治Segment\_Tree\_Partition}
\inputminted{c++}{../离线算法/线段树分治Segment_Tree_Partition.cpp}

\newpage
\section{杂项}
\subsection{快读快写Fast\_Read\&Write}
\inputminted{c++}{../杂项/快读快写Fast_Read\&Write.cpp}
\subsection{快速幂Fast\_Power}
\inputminted{c++}{../杂项/快速幂Fast_Power.cpp}
\subsection{矩阵快速幂Matrix\_Fast\_Power}
\inputminted{c++}{../杂项/矩阵快速幂Matrix_Fast_Power.cpp}
\subsection{矩阵加速Matrix\_Acceleration}
\inputminted{c++}{../杂项/矩阵加速Matrix_Acceleration.cpp}
\subsection{最长公共子序列Longest\_Increasing\_Subsequence}
\inputminted{c++}{../杂项/最长公共子序列Longest_Increasing_Subsequence.cpp}
\subsection{模拟退火Simulated\_Annealing}
\inputminted{c++}{../杂项/模拟退火Simulated_Annealing.cpp}
\subsection{快速沃尔什变换Fast\_Walsh\_Transform}
\inputminted{c++}{../杂项/快速沃尔什变换Fast_Walsh_Transform.cpp}
\subsection{快速莫比乌斯变换Fast\_Mobius\_Transform}
\inputminted{c++}{../杂项/快速莫比乌斯变换Fast_Mobius_Transform.cpp}
\subsection{快速子集变换Fast\_Subset\_Transform}
\inputminted{c++}{../杂项/快速子集变换Fast_Subset_Transform.cpp}
\subsection{24点24\_Point}
\inputminted{c++}{../杂项/24点24_Point.cpp}




\end{document}