\subsubsection{第一类斯特林数}

(斯特林轮换数)

$\begin{bmatrix}n\\ k\end{bmatrix}$ 表示将 $n$ 个两两不同的元素,划分为 $k$ 个非空圆排列的方案数。

递推式

$$
\begin{bmatrix}n\\ k\end{bmatrix}=\begin{bmatrix}n-1\\ k-1\end{bmatrix}+(n-1)\begin{bmatrix}n-1\\ k\end{bmatrix}
$$

边界是 $\begin{bmatrix}n\\ 0\end{bmatrix}=[n=0]$ 。

\subsubsection{第二类斯特林数}

(斯特林子集数)

$\begin{Bmatrix}n\\ k\end{Bmatrix}$ 表示将 $n$ 个两两不同的元素,划分为 $k$ 个非空子集的方案数。

递推式

$$
\begin{Bmatrix}n\\ k\end{Bmatrix}=\begin{Bmatrix}n-1\\ k-1\end{Bmatrix}+k\begin{Bmatrix}n-1\\ k\end{Bmatrix}
$$

边界是 $\begin{Bmatrix}n\\ 0\end{Bmatrix}=[n=0]$ 。

\subsubsection{上升幂与普通幂的相互转化}

我们记上升阶乘幂 $x^{\overline{n}}=\prod_{k=0}^{n-1} (x+k)$ 。

则可以利用下面的恒等式将上升幂转化为普通幂:

$$
x^{\overline{n}}=\sum_{k} \begin{bmatrix}n\\ k\end{bmatrix} x^k
$$

如果将普通幂转化为上升幂,则有下面的恒等式:

$$
x^n=\sum_{k} \begin{Bmatrix}n\\ k\end{Bmatrix} (-1)^{n-k} x^{\overline{k}}
$$

\subsubsection{下降幂与普通幂的相互转化}

我们记下降阶乘幂 $x^{\underline{n}}=\dfrac{x!}{(x-n)!}=\prod_{k=0}^{n-1} (x-k)$ 。

则可以利用下面的恒等式将普通幂转化为下降幂:

$$
x^n=\sum_{k} \begin{Bmatrix}n\\ k\end{Bmatrix} x^{\underline{k}}
$$

如果将下降幂转化为普通幂,则有下面的恒等式:

$$
x^{\underline{n}}=\sum_{k} \begin{bmatrix}n\\ k\end{bmatrix} (-1)^{n-k} x^k
$$